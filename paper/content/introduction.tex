% !TEX root = ../thesis-example.tex
%
\chapter{Introduction}
\label{sec:intro}

%\cleanchapterquote{}{}{(TODO)}

Accessibility is a key assistive accommodation to enable people  overcome physical, technological, or informational barriers \cite{acessguide13}.
As companies seek for cost reductions, revenue opportunities, but yet with good quality of customer service, interactive kiosks has been a leading  medium for communication trend to enable their customers with self-service options \cite{abi16}.
A \emph{interactive kiosk} is a computation system terminal with spcialized hardware and software designed for self-service options, such as information, communication, commerce, entertainment, and education \cite{satish2012money}.
As a computation system, a interactive kiosk systems has specific constrains associated with its components.
For instance, power constrains applies since the terminal usually is implemented in low power devices.
In this report we present the proposal and implementation of a low cost interactive kiosk with voice recognition feature.


In our project, we sought for tools to enhance the experience and quality on a interactive kiosk systems.
\emph{Wit.ai} \cite{1_wit.ai_2016} is a cloud service that allows a simple implementation of natural language processing for developers.
With only a few lines of its code, Wit.ai let developers build a speech recognition and voice control console.


As for an efficient system to manage the data and improve the information retrieval, the implementation was executed with the open-source web platform called Meteor.js.
With \emph{Meteor.js} \cite{meteor}, or Meteor for short, developers can create applications using the Web 2.0 paradigms.
The platform provide a reactive approach by focusing on the data flow.
This is done by creating and managing events in the application.
Also, data management is done with a non-SQL database called \emph{MongoDB} \cite{mongo}.
Meteor simplify the application development by providing an unique programming language, javascript, throughout the whole stack process.
Moreover, the platform makes available a set of common tools for business logic and data management.
Finally, Meteor deploys the application in desktop and mobile without needing to change the source code.


The cost of the project is manly dictated by the hardware component that holds the system.
Some platforms often utilized for low cost system are BeagleBone Black,  Arduino,  Phidgets, Udoo, and Electric Imp \cite{kubitza2013ingredients,pi2012raspberry}.
In our project, the hardware chosen \emph{Raspberry Pi} \cite{pi2012raspberry}, a low cost single-board computer that runs on the various distribution of Linux operating system and can be programmed as needed.
One of the Raspberry Pi advantages is it capability for a remote communication, making it a suitable choice for applications in the \emph{Internet of Things} (IoT) concept.


With the implementation some issues were faced.
For instance, the initial plan for our final project included a service inside the app for indoor navigation.
This feature is commonly seen in big centres such malls, hospitals or airports.
However, indoor navigation itself is a complex and well known task problem in computer science and engineering.
Although some systems are available, such as \emph{OpenStreetMap} and \emph{Anyplace} \cite{zeinalipour:IEEEIC16}, they are not at low level for simple systems and not suitable for small projects. 
In this way, we wanted to show how resource constrain devices can still be used to provide such service by using cloud computing.
After facing those constrains (better described in Chapter \ref{sec:ex-project}), we then turned our attentions for the software design and search engine of the interactive kiosk.
We implemented basic methods for searching graphs that perform efficiently search on a tree structure: \emph{depth first search} and the \emph{breadth first search} \cite{joyner2010algorithmic}.

The project presents many salient features.
First, it can be voice activated and has state-of-the art speech recognition throughout the usage of the Wit.ai.
Second, it performs an solid search on the tree structure implemented on the web platform Meteor.js.
Features as uncertainty decision and incomplete information are well handled by the system.
Third, the hardware of implementation was on a Raspberry Pi, which is well suited for the project since it is (i) low cost system, (ii) customizable and (iii) programmable.
The result is an interactive kiosk that ``understands'' voice requests of informations, answering them with a variety of possibilities responses, all editable by the administrators.

The reminder is as follows.
Chapter \ref{sec:back} shows a quick overview on the history and background informations of interactive kiosk systems, reactive programming, and graph data structures.
The project implementation along with related works, platforms used, and results are given in Chapter \ref{sec:project}.
Chapter \ref{sec:ex-project} shows the discussion on constrains of developing an the app for indoor navigation.
Conclusions are given in Chapter \ref{sec:conclusion}.
Additional and complementary coding are provided in Appendix Chapter \ref{sec:appendices}.

