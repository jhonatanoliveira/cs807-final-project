\chapter{Conclusion}
\label{sec:conclusion}

%\cleanchapterquote{TODO.}{TODO}{(TODO)}

An interactive kiosk contains specialized hardware and software designed within a public exhibit.
As a computation system, a interactive kiosk systems has specific constrains associated with its components.
Thus, an interactive kiosk must address resourceful solutions to minimize its possible constrains.
In this project report we have shown an interactive kiosk with speech recognition for academic environments.
We utilized three different platforms, including two cloud computing ones, namely Wit.ai and Meteor.js, and one hardware, Raspberry Pi.

Our focus with the project was to use clouding platforms to overcome hardware platform limitations.
For such, spcialized concepts were required as reactive programming and graph data structures.
In this way, we could make a system that interact with the user by a terminal client using a Raspberry Pi, and replies to the user by processing the query on the server side.
These services are of two types: data services and external services.
The data service is internally described by a knowledge tree.
All corresponding knowledge trees can be edited in this same administrator screen.
Basic methods for searching graphs such as DFS were essential for traversing the tree for proper information retrieval.

The initial plan, however, included a service inside the app for indoor navigation.
Indoor navigation itself is a complex and well known task problem in computer science and engineering.
We did a broad research on publicly available solutions for indoor navigation, including paid, free or open source ones.
After several tests, we decided to do not go further with the indoor navigation due to its complexity requirements for implementation.
Even though this application were not able to be applied, we achieved our goal of implementing a self-service kiosk (university level).