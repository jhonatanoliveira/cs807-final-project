% !TEX root = ../thesis-example.tex
%
\pdfbookmark[0]{Abstract}{Abstract}
\chapter*{Abstract}
\label{sec:abstract}
\vspace*{-10mm}


An interactive kiosk is a computer terminal that provides access to information and applications for communication.
It contains specialized hardware and software designed within a public exhibit.
Thus, an interactive kiosk must address resourceful solutions to minimize its possible constrains.
In this project report we propose an interactive kiosk with speech recognition for academic environments.
We utilized \emph{Wit.ai}, a \emph{cloud service} that turns speech or text into actionable data.
Wit works on the project as a accessibility tool an interactive feature.
The server, data modeling to the client and user interface was built with \emph{Meteor.js}, a full-stack javascript framework.
The hardware of chosen was Raspberry Pi 2, a low cost single-board computer with salient features to support efficiency and reliability for the project interactive kiosk.

In this project report we discuss the process and constrains to build an interactive kiosk with speech recognition.
We describe the implementation and functionality of the project to use clouding platforms to overcome hardware platform limitations.
We utilized a system implemented in Meteor.js containing a server and a client.
The server, a more powerful computer, is used as a central processing unit and only forward responses for the client's processing requests.
Finally, we discuss the challenges of localization and map apps that impose barriers efforts for simple implementations.